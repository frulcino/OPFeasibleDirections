\documentclass{article}
\usepackage{amsmath} %math fonts
\usepackage{mathtools} %align and coloneqq...

\usepackage{adigraph} %to draw graphs
\usepackage{xcolor}
\usepackage{todonotes}
\newcommand{\ambrogio}[2][]{\todo[color=violet!40!,#1]{\textsf{VW:} #2}}
\newcommand{\gr}[2][]{\todo[color=green!20,#1]{\textsf{G:} #2}}


%theorem enviroments:
\newtheorem{example}{Example}
\newtheorem{assumption}{Assumption}
\newtheorem{definition}{Definition}
\newtheorem{theorem}{Theorem}
\newtheorem{corollary}{Corollary}[theorem]
\newtheorem{lemma}[theorem]{Lemma}
\newtheorem{prop}[theorem]{Proposition}
\newtheorem{observation}[theorem]{Observation}
\newtheorem{problem}{Problem}


\newcommand{\nc}{\newcommand}
\nc{\st}{dx} %step symbols
\nc{\stc}{dc}
\nc{\sts}{ds}
\nc{\stP}{dP}
\nc{\stQ}{dQ}
\nc{\opfcost}{F} %cost of generation
\nc{\stS}{dS} %forse avrebbe avuto più senso dare un input ma ok...
\nc{\boB}{{\mathbf{B}}}
\nc{\boL}{{\mathbf{L}}}
\nc{\boY}{{\mathbf{Y}}}
\nc{\boI}{{\mathbf{I}}}
\nc{\boV}{{\mathbf{V}}}
\nc{\boS}{{\mathbf{S}}}
\nc{\boG}{\mathcal{G}}
\nc{\cN}{\mathcal{N}}


\begin{document}

\section{Random ideas}


\section{Jabr Model}



For the OPF model construction we the network as directed graph
$(\boB,\boL)$ where $\boB$ is the set of Buses and and $\boL \subset \boB \times \boB$ is the 
set of branches of the network and for each adjacent buses $k,m$ both $(k,m)$ and $(m,k)$ are
in $\boL$. So the line $l$ adjacent to $k,m$ is modeled by two edges in the arc $\{(k,m),(m,k)\}$.
$L$ can be partitioned in $\boL_0$ and $L_1$ with $|L_0|=|L_1|$ where every line $l$, adjacent 
to the buses $k,m$ and with a transformer at $k$, is oriented so that $(k,m) \in L_0$ and 
$(m,k) \in \boL_1$. We also consider a set $\boG$ of generators, partitioned into 
(possibly empty) subsets $\boG_k$ for every bus $k \in \boB$.
We consider the following convex Jabr relaxation of the OPF problem:
\begin{align}
  \label{Jabr equality model}
  \inf_{\substack{P^G_g, Q^G_g, c_{km}, \\ s_{km}, S_{km}, P_{km}, Q_{km}}} & \opfcost(x) \coloneqq \sum_{g \in \mathcal{G}}{F_g(P_g^G)} \\
  \text{Subject to:} & \; \forall km \in \boL \nonumber \\
  & c_{km}^2 + s_{mk}^2 \leq c_{kk}c_{mm} \quad \text{Jabr constraint} \label{Jabr constraint}\\
  & P_{km} = G_{kk}c_{kk}+G_{km}c_{km}+B_{km}s_{km} \label{c to P} \\
  & Q_{km} = -B_{kk}c_{kk}-B_{km}c_{km}+G_{km}s_{km} \label{c to Q}\\
  & S_{km} = P_{km} + jQ_{km} \label{PQ to S}\\
  & \text{Power balance constraints:}\; \forall k \in \boB\nonumber \\
  & \sum_{km \in L}S_{km}+P_k^L+iQ_k^L = \sum_{g \in \mathcal{G}(k)}{P_g^G} + i\sum_{g \in \mathcal{G}(k)}{Q_g^G} \label{constr: Power Flow Constraint J}\\
  & \text{Power flow, Voltage, and Power generation limits:} \nonumber \\
  & P_{km}^2 + Q_{km}^2 \leq U_{km} \label{Power Flow Magnitude Constraint J}\\
  & V_k^{\text{min}^2} \leq c_{kk} \leq V_k^{\text{max}^2} \label{Voltage Magnitude Constraint J} \\
  & P_g^{\text{min}} \leq P_g^G \leq P_g^{\text{max}}\label{Power generation Magnitude Constraint J} \\
  & c_{kk} \geq 0 \; \\ 
  & c_{km} = c_{mk}, \;s_{km} = - s_{mk}.
  \end{align}

This relaxation is in general not exact. We can recover exactness thanks to the following result:
\begin{prop}
  Model \eqref{Jabr equality model} with the additional \emph{loop constraint} \eqref{loop constraint} for every loop in a cycle basis of \((\boB,\boL)\) is exact, we refer to this new model as the \emph{Exact Jabr formulation}

  \begin{equation}
    \label{loop constraint}
    \sum_{k = 0}^{\lfloor n/2 \rfloor}\sum_{\substack{A \subset [n]\\|A|=2k}}(-1)^k\prod_{h \in A}s_{k_hk_{h+1}}\prod_{h \in A^c}c_{k_hk_{h+1}}=\prod_{k=1}^nc_{k_i,k_i}.
  \end{equation}
  \end{prop}
This result suggests the following approaches to either find a feasible solution or move along the space of feasible solutions.
\begin{itemize}
  \item Such relaxation is exact on tree Networks (also known as radial networks). Our objective is, given a network \(\cN = (\boB,\boL)\) which can also not be a tree, consider a radial subnetwork \(\cN' = (\boB,\boL')\), with \(\boL' \subset \boL \) and consider the Jabr model  on \(\cN'\).
  This solution is not necessarily feasible for the original problem \(\cN\), our objective is to iteratively recover a feasible solution for \(\cN\). 
  \vspace{1cm}
  Since the Jabr relaxation is exact on \(\cN'\) it follows that the constraints \ref{Jabr constraint} are respected, the constraints which are violated are the flow constraints on the leaves. We can try to recover feasibility my moving along the solution to the Jabr and Loop constraints.
  \item Given a feasible solution, find feasible directions.
\end{itemize}

\section{Feasible directions}

Let \(x_0 = (P_0,Q_0,c_0,s_0)\) be a feasible solution of the OPF problem \eqref{Jabr equality model} with the loop constraints \eqref{loop constraint}.
We want to find feasible directions \(x_0 = (\stP,\stQ,\stc,\sts)\), that is such that \(x_1 = (P_0+\stP,Q_0+\stQ,c_0+\stc)\) is still a feasible solution of the OPF problem.
We consider each constraint of the Exact Jabr Formulation separately do get feasible directions.

\subsection{Jabr Constraint}

Since \(x_0\), the jabr equality holds: \(c_{ii}c_{jj}= c_{ij}^2+s_{ij}^2\).
Adding the movement \(\st\) we want that \((c_{ii}+\stc_{ii})(c_jj+\stc_{jj}) = (c_{ij}+\stc_{ij})^2+(s_{ij}+\sts_{ij})^2\). By expanding the terms and considering that the equality holds for \(x_0\) this is equivalent to:
\begin{equation}
  \stc_{ii}\stc_{jj}+2c_{ii}\stc_{jj}+2\stc_{ii}c_{jj}= \stc_{ij}^2+c_{ij}\stc_{ij}+\sts_{ij}^2+s_{ij}\sts_{ij}
\end{equation}
For now we consider movements where \(\stc_{ii}\) is not zero only on an independent set of nodes in the graph, this way the constraint simplifies to: 
\begin{equation}
  \stc_{ij}^2+c_{ij}\stc_{ij}+\sts_{ij}^2+s_{ij}\sts_{ij}-2\stc_{ii}c_{jj} = 0 
\end{equation}
The solutions to this constraint can be found by minimizing the following minimization problem:
\begin{equation}
  \min  (\stc_{ij}^2+c_{ij}\stc_{ij}+\sts_{ij}^2+s_{ij}\sts_{ij}-2\stc_{ii}c_{jj})^2
\end{equation}
Which can be solved by gradient descent methods. Since we want to solve this for all \((i,j)\in \boL\), we instead solve the following:
\begin{equation}
  \min  CJ(\stc,ds,\stP) = \sum_{(i,j)\in \boL} (\stc_{ij}^2+c_{ij}\stc_{ij}+\sts_{ij}^2+s_{ij}\sts_{ij}-2\stc_{ii}c_{jj})^2
\end{equation}
\gr[inline]{Anche questo può essere risolto con metodo gradiente? Sappiamo che il minimo globale è zero, ma possono esserci minimi locali? Dobbiamo per forze imporre gli spostamenti solo su nodi indipendenti?}


\subsection{Loop Constraints}
Unfortunately, the number of monomials to computer for checking the various of loop constraints grows factorially with the length of the cycle. Thus trying to apply a similar method to find feasible directions satisfying the loop constraints as we did for the Jabr constraints would be computationally intractable.
We can try the following approaches:
\begin{itemize}
  \item \textcolor{gray}{Find the cycle basis with minimal weight}
  \item Each cycle can be broken down by adding auxiliary edges in the Network, decomposing each loop constraints in loop contraints associated to cycle of the desired length. For each auxiliary edge \(e = (e_0,e_1)\) we add the auxiliary variable \(c_e,s_e\), the associated Jabr constraints: \(c_e^2 + s_e^2 = c_{e_0}c_{e_1}\). We can now divide the cycles containing \(e_0\) and \(e_1\) in two cycles containing the edge \(e\) and thus obtaining smaller associated loop contraints.
  \item Loop constraints of cycles of length equal to 3 or 4 can be rewritten as 2 polynomial constraints of degree 2.
  
\end{itemize}

Let us consider the loop constraints on the following cycle:
\NewAdigraph{threecycle}{
  1:2;
  2:2;
  3:2;
}{
  1,2;
  2,3;
  3,1;
}[-]
\begin{center}

\threecycle{}
\end{center}
This corresponds to the following polynomial constraints:
\begin{equation}
  p_3 \coloneqq c_{12}(c_{23}c_{31} - s_{23}s_{31})-s_{12}(s_{23}c_{31}+c_{23}s_{31})-c_{11}c_{22}c_{33}
\end{equation}
We also define the following polynomials associated to the Jabr constraints at each edge:
\begin{equation}
  p_{ij} \coloneqq c_{ij}^2+s_{ij}^2-c_{ii}c_{jj}
\end{equation}
and
\begin{align}
  q_3^1 &= s_{12}c_{33} + c_{23}s_{31} + s_{23}c_{31} \label{constr: loop decomposition 1}\\
  d_3^2 &= c_{12}c_{33} + c_{23}c_{31} + s_{23}s_{31} \label{constr: loop decomposition 2}
\end{align}
Then, we have the following results which let's us to rewrite the constraints associated to a cycle of length three as two polynomial constraints of degree 2.
\begin{prop}
  \begin{equation}
    \{(c,s)\mid p_3 = p_{12} = p_{23} = p_{31} = 0\} = \{(c,s)\mid q_3^1 = q_3^2 = p_{12} = p_{23} = p_{31} = 0\} 
  \end{equation}
\end{prop}

If we don't restrict the possible direction \(\st=(d_c,d_s,d_P)\), imposing that constraints \eqref{constr: loop decomposition 1}, \eqref{constr: loop decomposition 2} hold for \(x + \st\) would impose polynomial constraints on \(\st\).
Alternatively, since  for example \(q_3^1\) is a bilinear polynomial respect to the vectors \(v_1=(s_{12}, c_{23}, s_{23})\) and \(v_2=(c_{33}, s_{31}, c_{31})\), we can fix the variables in \(v_2\), that is \(dv_2 = 0\) and move along \(v_1\).
This way constraint \(q_3^1=0\) imposes a linear constraint on \(d_x\). 
We can proceed on a similar way for \(q_3^2\). We note that one can consider instead to move along \(v_2\) and that there are other possible choices for \(v_1\) and \(v_2\). 
\gr[inline]{The hope is that even though we are restricting the directions for a single step, that by concatenating steps we can obtain a "good" amount of feasible directions.}
\gr[inline]{TODO: think well about possible directions}

\subsection{Flow constraints}
The flow constraint \eqref{constr: Power Flow Constraint J} also induce linear constraints on the step \(\st\):
\begin{align}
  \stP_{km} &= G_{kk}\stc_{kk}+G_{km}\stc_{km}+B_{km}\sts_{km}\\
  \stQ_{km} &= -B_{kk}\stc_{kk}-B_{km}\stc_{km}+G_{km}\sts_{km} \\
  \stS_{km} &= \stP_{km} + j\stQ_{km} \\
  \sum_{km \in L}\stS_{km}+\stP_k^L+i\stQ_k^L &= \sum_{g \in \mathcal{G}(k)}{\stP_g^G} + i\sum_{g \in \mathcal{G}(k)}{\stQ_g^G} \label{Power Flow Constraint J} 
\end{align}


\subsection{Other constraints}
Magnitude constraints are what limit the step size:
\begin{align}
  V_k^{\text{min}^2} \leq c_{kk} + \stc_{kk} \leq V_k^{\text{max}^2}  \\
  P_g^{\text{min}} \leq P_g^G + \stP_g^G\leq P_g^{\text{max}}\\\
  c_{kk} + \stc_{kk} \geq 0 \;
\end{align}
\gr[inline]{what about this one?}
\begin{equation*}
  P_{km}^2 + Q_{km}^2 \leq U_{km}
\end{equation*}

\subsection{Step Problem}
Combining the considerations in the previous subsections we obtaing the following problem to calculate a feasible step.
Where given a graph, we find a cycle basis, we decompose each cycle in a 3/4-cycles, for each of these smaller cycles we consider their loop constraints and pick \gr[]{how?} which variables do move in the step \(\st\).
\gr[inline]{Da riscrivere meglio, i vincoli su \(q_e^1,q_e^2\) sono per ciascun arco in un ciclo in una base di cicli, ma se lo stesso arco è in due basi di cicli diverse, se i due cicli non hanno nodi in comune allora sono due vincoli diversi}

\begin{align}
  \label{prob: Step Problem}
  \min  CJ(\st) = & sum_{(i,j)\in \boL} (\stc_{ij}^2+c_{ij}\stc_{ij}+\sts_{ij}^2+s_{ij}\sts_{ij}-2\stc_{ii}c_{jj})^2 \\
  \text{Subject to:} \nonumber & \\
  q_e^1 &= \sts_{12}c_{33} + \stc_{23}s_{31} + \sts_{23}c_{31}\\
  q_e^2 &= \stc_{12}c_{33} + \stc_{23}c_{31} + \sts_{23}s_{31} \\
  \stP_{km} &= G_{kk}\stc_{kk}+G_{km}\stc_{km}+B_{km}\sts_{km}\\
  \stQ_{km} &= -B_{kk}\stc_{kk}-B_{km}\stc_{km}+G_{km}\sts_{km} \\
  \stS_{km} &= \stP_{km} + j\stQ_{km} \\
  \sum_{km \in L}\stS_{km} &= \sum_{g \in \mathcal{G}(k)}{\stP_g^G} + i\sum_{g \in \mathcal{G}(k)}{\stQ_g^G} \\
  &V_k^{\text{min}^2} \leq c_{kk} + \stc_{kk} \leq V_k^{\text{max}^2}  \\
  &P_g^{\text{min}} \leq P_g^G + \stP_g^G\leq P_g^{\text{max}}\\\
  &c_{kk} + \stc_{kk} \geq 0 \;
\end{align}

If the optimal solutions of problem \eqref{prob: Step Problem} are feasible steps if and only if their cost is 0. 
Leaven like this one obvious solution is \(dx = 0\). Possible ideas.
\begin{itemize}
  \item Impose that the generation cost must decrease, by putting the scalar product of \(\st\) with the gradient of the cost function \(\opfcost\) in \(x\) to be smaller than 0. \(\st \nabla F(x) < 0 \). With this additional constraint if it's an equality then the family of the possible direction doesn't have a decreasing cost direction.
\end{itemize}
Since the constraints are affine, problem \eqref{prob: Step Problem} can be solved with a projected gradient descent algorithm.
\gr[]{If we don't consider magnitude constraints, we can consider directly the projection of the gradient on the subspace of feasible solutions, this would be easier than considering the projection on a generic poligon (I think)}


\end{document}